\documentclass{article}
\usepackage[utf8]{inputenc}
\usepackage[T1]{fontenc}
\usepackage{lmodern}
\usepackage{amsmath,amssymb}
\usepackage{graphics,url}

\title{CIS099, Intelligent Game Agents -- Project report}
\author{Ryan Gormley, Eliot Kaplan, Fabian Peternek \\ Teamname: Team BWAAAAAAMP}

\begin{document}
\maketitle
\section*{Prelude}
The following report shall describe our efforts to create an artificial intelligence, that conquers
the world as known to the ants or at least wins the competition. We will start by an explanation of
our first plan and then detail the method we used in the end.
\section*{Act I: The plan}
The basic idea was the following: We realised, that reinforcement learning works well if all the
parameters are set correctly. Setting those parameters however is quite difficult, without some
deeper insight into the problem. This is especially true for the reward function, where one reward
that is out of line can unintentionally bias the learning process into one undesired direction. We
therefore wanted to try the following: We would use reinforcement learning (or rather $q$-learning)
to train the bot, but we would first learn the reward function by using another search strategy. To
do this we first intended to use a genetic algorithm approach, but this would have taken too long to
yield a good result. A good reward function was therefore to be found by grid search.
\section*{Act II: The approach}
The first step in getting this plan to work was to create features, that incorporate the anthills,
which were previously not used. We therefore created features for the distance of an ant to the
closest known own and enemy ant hills, and how much food is stored in the hill. The latter one
however is only an approximation, as the game unfortunately does not communicate this number to the
bots. Furthermore the $q$-learning algorithm was adapted to get a parameterized input for the reward
function and the other important variables and a trainer for this parameters was created using a
grid search approach.
\section*{Act III: The problems}
At this point we could start to train our bots, but unfortunately had to realize, that the training
does not work. What went wrong? Likely more than one thing, but the most important one is: The game
has changed considerably. The fact, that the only way to get points is now to destroy an enemy ant
hill makes it difficult to evaluate the results of training, where we used the difference in score
to discern, which bot was better. However as long as no bot accidentally moved onto an enemy hill,
there was no way to tell, if one bot did better than another. At this point ``won the most games''
might have been a better approach of evaluation.

Another problem is, that our reward function had no information about the hills. We could define a
negative reward for loosing the own ant hill, but we did not think of incorporating the destroying
of enemy hills into our reward function.
\section*{Act: IV: The solution}
Due to the changed scoring we could not get the idea of searching for a good reward function to
work in time. We therefore decided to tailor the rewards by hand again and run the $q$-learning on
that. However we did think of rewarding the ants for destroying enemy hills this time. This approach
finally lead to a working bot.
\section*{Act V: The Tournament}
As our bot was finished just in time for the final tournament, we could not train a lot.
Furthermore, the tournament would be our evaluation of the bot. Considering those prerequisites it
actually did not do bad. It lost one game and almost won another one, only loosing because it timed
out. With some more tweaking we are quite confident, that this bot would be working pretty well.
\section*{Epilogue}
The last two pages detailed our ants way to world domination, even though it did not quite manage to
get there. We described our original plan for the bot, which parts worked and which parts did not
work. We supplied some some assumptions why these parts did not work and then detailed how we solved
the problems in time. The report is concluded by a short evaluation of its performance, showing,
that the bot was not terrible, even though some more tweaking would be pretty helpful.
\end{document}
